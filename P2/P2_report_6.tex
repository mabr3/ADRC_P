
%----------------------------------------------------------------------------------------
%	PACKAGES AND OTHER DOCUMENT CONFIGURATIONS
%----------------------------------------------------------------------------------------

\documentclass[twocolumn]{article} % A4 paper size, default 11pt font size and oneside for equal margins
\usepackage[margin = 1in]{geometry}
\usepackage[portuguese]{babel}
\usepackage{algpseudocode} %para pseudo código
\usepackage{algorithm}

\usepackage[utf8]{inputenc} % Required for inputting international characters
\usepackage[T1]{fontenc} % Output font encoding for international characters
\usepackage{fouriernc} % Use the New Century Schoolbook font
\usepackage{graphicx}

%----------------------------------------------------------------------------------------
%	TITLE PAGE
%----------------------------------------------------------------------------------------

\begin{document} 

\begin{titlepage} % Suppresses headers and footers on the title page

	\begin{figure}[t]
	\graphicspath{ {../../} }
	\includegraphics[width = 8 cm]{logo.png}
	\centering
	\end{figure}
	
	\centering % Centre everything on the title page
	
	\scshape % Use small caps for all text on the title page
	
	\vspace*{\baselineskip} % White space at the top of the page
	
	%------------------------------------------------
	%	Title
	%------------------------------------------------
	
	\rule{\textwidth}{1.6pt}\vspace*{-\baselineskip}\vspace*{2pt} % Thick horizontal rule
	\rule{\textwidth}{0.4pt} % Thin horizontal rule
	
	\vspace{0.75\baselineskip} % Whitespace above the title
	
	{\LARGE 2\textsuperscript{\underline{o}} Projeto:\\Static Analysis of Inter-AS Routing\par} % Title
	
	\vspace{0.75\baselineskip} % Whitespace below the title
	
	\rule{\textwidth}{0.4pt}\vspace*{-\baselineskip}\vspace{3.2pt} % Thin horizontal rule
	\rule{\textwidth}{1.6pt} % Thick horizontal rule
	
	\vspace{2\baselineskip} % Whitespace after the title block
	
	%------------------------------------------------
	%	Subtitle
	%------------------------------------------------
	
	 % Subtitle or further description
	Algoritmia e Desempenho em Redes de Computadores \\ Docente: João Sobrinho

	\vspace*{3\baselineskip} % Whitespace under the subtitle
	
	%------------------------------------------------
	%	Editor(s)
	%------------------------------------------------
	
	
	\vspace{0.5\baselineskip} % Whitespace before the editors
	
	{\scshape\Large Miguel Rodrigues N\textsuperscript{\underline{o}} 76176 \\ Pedro Esteves N\textsuperscript{\underline{o}} 77060 \\} % Editor list
	Grupo 6 \\
	
	\vspace{0.5\baselineskip} % Whitespace below the editor list
	
	\vfill % Whitespace between editor names and publisher logo
	
	\vspace{0.3\baselineskip} % Whitespace under the publisher logo
	
	17/11/2017 % Publication year

\end{titlepage}

%\shipout\null %Blank Page without increasing page number -para imprimir. comentar na versão para entregar.

%----------------------------------------------------------------------------------------

	
	\section{Introdução}

		Neste segundo mini-projeto são explorados os conceitos de \emph{routing}, fazendo uma análise estática de uma rede de \emph{Autonomous Systems} (AS), simulando o fucnionamento de uma internet e seguindo, de uma forma aproximada, as preferências e regras do protoco BGP (\emph{Border Gateway Protocol}). Através da leitura de um ficheiro de texto que indica as diferentes ligações entre os diferentes nós (ASes), temos como objetivo criar o grafo que a representa e realizar uma análise das suas propriedas.\\
		\noindent Em primeiro lugar, é feita a verificação da existência de ciclos de clientes, isto  é, se é possível chegar de um nó a si mesmo novamente, mas seguindo apenas ligações do tipo cliente (\emph{Costumer}). Tal não pode acontecer numa rede como esta.\\
		\noindent Depois, é feita uma análise quanto à sua conectividade: ver se é \emph{Comercialmente Conexa}, isto é, se partindo de um nó é possível chegar a qualquer outro, respeitando as regras referentes aos caminhos possíveis. Numa internet tem de se verificar esta conetividade.\\
		\noindent Por fim, é necessário verificar quais os caminhos eleitos por todos os nós para chegar a todos os outros nós e produzir estatísticas relativas a esta rede.


	\section{Funções e Estruturas}
		\subsection{Estruturas}
 
			Para representar o grafo, ou seja, a rede de ASes e repetivas ligações, é usado um vetor de ponteiros para nós (\emph{Nodes}), inicializado com 70000 posições, pois temos a informação de que o número não será superior a este (será de aproximadamente $2^{16}$ = 65536), não sendo necessário fazer alocação dinâmica neste caso. Cada uma destas posições apresenta para um nó em específico, cujo identificador é um número (neste caso entre 0 e 70000) e que contem várias informações sobre a AS em questão, como o número de \emph{Providers} (P), de \emph{Peers} (R) e de \emph{Costumers} (C). Em conjunto com as várias informações relevantes da AS, existem três apontadores para listas de adjacência, uma para cada tipo de ligação.\\
			\noindent Estas listas de adjacência vão conter estruturas do tipo \emph{NodeAdj} que guardam apenas o o ponteiro para a próxima estrutura (ou seja, o próximo vizinho do tipo repetivo à lista de adjacência: C, P ou R) e o número do identificador da AS.

		\subsection{Leitura da Rede}

			De forma a criar a rede, é lido o ficheiro de texto correspondente linha a linha e é feita a alocação dos nós, alocando a AS que é a \emph{tail}, inicializando os vários parâmetros desse nó (como o número de \emph{peers}, de \emph{providers},etc) e a respetiva ligação. Como se sabe que as ligações são dadas de forma correcta e que para cada ligação P é dada uma C e que para cada ligação R é dada outra R, não há problema em alocar apenas uma das AS que aparecem na linha lida. Tendo em conta o tipo de ligação entre as ASes, é criado um nó no início da lista de adjacência correspondente da \emph{tail} dessa ligação.\\

			\noindent Para criar a ligação é usada a função \emph{AddEdge} que, recebendo o nó de onde a ligação provém, cria essa aresta para a \emph{head} da ligação e insere-a na respetiva lista de adjacência, tendo em conta o tipo de ligação e a informação do nó de destino.No final da alocação são percorridos todos os nós de forma a obter a lista de \emph{Tier1s} (T1s), que será útil nas restantes funções.

			\begin{algorithm}[htbp]
			\caption{AddEdge}
			\begin{algorithmic}[1]
				
				\State type := Tipo de Ligação
				\State Aloca nó auxiliar Aux

				\If{type == 1}
					\State Ligação C
					\State No->N\_Costumers ++;
					\State Adiciona Aux a ListaAdj\_C
				\Else
					\If{type == 2}
						\State Ligação R
						\State No->N\_Peers ++
						\State Adiciona Aux a ListaAdj\_R
					\Else
						\State Ligação P
						\State No->N\_Providers ++
						\State Adiciona Aux a ListaAdj\_P
					\EndIf
				\EndIf

			\end{algorithmic}
			\end{algorithm}

			\noindent Para que não existam problemas de memória são criadas as funções \emph{FreeGraph} e \emph{FreeAdjList}, que vão percorrer todos os nós do grafo e todas as listas de adjacência desses nós, libertando-os. Assim, essa libertação tem uma complexidade de $\mathcal{O}$(n+m), onde n são os nós e m as arestas. Também a inserção terá esta complexidade.

		\subsection{Verificação de Ciclo de Clientes}

			Para fazer a verificação da existência de ciclos de clientes, são usadas pesquisas em profundiade (DFS - \emph{Depth First Search}), partindo de cada um dos \emph{Tier1s} e usando apenas ligações \emph{Costumer}. Quando uma AS for visitada, é usada uma variáve (\emph{visit}) que não permite que seja visitada outra vez, pois se já foi visitada e não voltou a si mesma então não faz parte de um ciclo, logo não é necessário voltar a entrar nesta AS. Por outro lado, sempre que uma AS faz aprte do caminho que está a ser percorrido pela DFS, tem ativa uma variável (\emph{ciclo}) que a identifica como parte do ciclo, se este existir. Assim, se se voltar a uma AS cuja variável \emph{ciclo} for 1, então é encontrado um ciclo e o programa pode encerrar. Sendo uma DFS, a sua complexidade é dada por $\mathcal{O}$(n+m), onde n são os nós e m as arestas, mas nem todas as arestas serão percorridas, pois basta entar numa AS uma vez e esta pode ter vários ligações até ela.

			\noindent As funções que fazem parte deste algoritmo são \emph{VerifyCycle}, que percorre a lista de T1s para chamar a função que percorre os nós (\emph{DFS}) e verificar se existe ciclo ou não. 

			\begin{algorithm}[htbp]
			\caption{DFS}
			\begin{algorithmic}[1]
				\State ciclo := 0 
				\State search := No->AdjList\_C
				\While{search != NULL}
					\If{ciclo == 1}
						\State Há um ciclo. Sair
					\Else
						\State Marcar nó como visitado.
						\State No->ciclo =1;
						\State DFS(Nó seguinte)
						\State No->ciclo = 0;
					\EndIf
					\State search := search->next

				\EndWhile
			\end{algorithmic}
			\end{algorithm}

		\subsection{Verificação de Conectividade}

			Tendo em conta as regras descritas no enunciado, para que uma rede seja Comercialmente Conexa, é necessário que todos os nós \emph{Tier1} sejam \emph{peers} uns dos outros, se não existirão ligações que não serão possíveis (por exemplo, um T1 A só pode chegar a um T1 C se existir um ligação entre eles do tipo R, visto que não podemutilizar várias ligações R nem usar ligações P após ligações C). No entanto, esta condição é, também, suficiente, pois todos os nós que não sejam \emph{Tier1s} vão ter pelo menos uma ligação do tipo P (e respetiva ligação C no sentido contrário), até que eventualmente se chegue a um T1 que, sendo \emph{peer} de todos os outros T1s, vai permitir que exista uma ligação até qualquer outro nó tipo. Se o nó em questão não tiver uma ligação P, isso faz dele um T1, o que leva a que tenha ligações R com todos os outros T1s, pela condição inicial. Sabendo que não existem ciclos de clientes pelo algoritmo anterior, também não existirão ciclos de fornecedor que possam fazer com que o que é aqui descrito não se aplique.\\

			\noindent Assim, basta verificar a existência de ligações \emph{peer} entre todos todos os \emph{Tier1s}. Tendo em conta que nos é dada a garantia de que as ligações no ficheiro de texto são apresentadas de forma correcta, basta verificar o número de ligações R que cada T1 tem com  outros T1s (pois não vão existir ligações repetidas ou em falta, devido à garantia de que as ligações são dadas de forma correcta).\\
			\noindent Como descrito no algoritmo seguinte, basta guardar a informação de todos os T1s (ou seja, número de ligações P é zero), contá-los e verificar que cada T1 tem N-1 ligações R com outros T1s, sendo N o número de T1s.\\


			\begin{algorithm}[htbp]
			\caption{VerifyCommerc}
			\begin{algorithmic}[1]
				
				\State T := Tier1 array
				\State N := Number of Tier1s

				\For{All T1 present in T}
					\State count := 0;
					\For{All Peers}
						\If{Peer is a T1}
							\State count ++; 
						\EndIf
					\EndFor
					\If{count != N-1}
						\State Not all T1s are Peers
						\State End
					\EndIf
				\EndFor
			\end{algorithmic}
			\end{algorithm}

			\noindent A complexidade deste algoritmo é de $\mathcal{O}$($n^{2}$), no entanto, n é o número de nós T1, que é um número bastante reduzido de nós, portanto não é problemático.d

		\subsection{Procura de Caminhos}
			De forma a saber todos os caminhos para todos os nós existentes na rede, é usado um algoritmo que recebe um nó de destino e calcula qual o tipo de caminho que cada nó vai ter para esse nó de destino, tendo em conta as prioridades de caminhos. Tendo em conta que é um elevado número de nós, o cálculo de todos os caminhos é um processo exigente.\\
			
			\noindent Sabendo que a rede é comercialmente conexa e que não existem ciclos de clientes, a primeira coisa a fazer é identificar todos os nós que terão caminhos C para o destino, ou seja, os nós que são \emph{Providers} do nó destino e os respetivos \emph{Providers} e assim sucessivamente. De seguida, os nós que vão ter ligações do tipo R são os nós que são \emph{peers} do nó de destino ou \emph{peers} dos nós que já apresentam ligações do tipo C. Todos os restantes nós serão caminhos P, daí a inicialização ser feita para este tipo de caminho, poupando tempo com a decisão do caminho para estes nós.\\

			\begin{algorithm}[htbp]
			\caption{Path}
			\begin{algorithmic}[1]
				
				\State Path := 3 para todos os nós
				\For{Todos os Providers do nó}
					\State Path := 1;
					\For{Todos os Peers do Provider}
						\If{Path != 1}
							\State Path = 2
						\EndIf
					\EndFor
					\State Repetir para todos os Providers dos Providers até ao T1
				\EndFor

			\end{algorithmic}
			\end{algorithm}

			\noindent Este algoritmo é repetido N vezes, onde N é o número de ASes existentes, de forma a obter estatísticas a apresentar sobre a rede toda.\\

			\noindent Usando este algoritmo, é possível saber quais são os tipos de caminhos a ser usados de forma eficiente, obtendo-os em menos de cinco minutos, sendo que o algoritmo apresenta uma complexidade de $\mathcal{O}$(blablabla) la blfdmdlf. No entanto, embora este algoritmo cumpra aquilo que é pedido no enunciado do projeto, para que fosse uma representação mais fiel do algoritmo BGP, ter-se-ia em conta, também, o número de hops de cada caminho, escolhendo sempre o caminho com menor número de hops (dentro do mesmo tipo de caminho).\\


		\subsection{Estatísticas}
			Através da análise feita tendo em conta apenas as preferências de caminho (C > R > P), pela análise da rede dada em \emph{LargeNetwork.txt}, contam-se 1,933,668,702 caminhos ($\approx$ $1.9\times 10$$^{9}$), sendo que estão divididos da forma apresentada na Tabela~\ref{Tabela1}.\\

			\begin{table}[ht]
				\caption{Distribuição dos caminhos na rede fornecida}
				\begin{center}
					\begin{tabular}{|c||c||c|}
						\hline
						Path Type &	\# de Paths & \% de Paths \\
						\hline
						C & 11,828,240 & 0.611699\\
						\hline
						R & 217,369,490 & 11.241299\\
						\hline
						P & 1,704,470,972 & 88.147001\\
						\hline
					\end{tabular}
				\end{center}
				\label{Tabela1}
			\end{table}

			\noindent É esperado que assim seja, pois há um número muito maior de \emph{stubs} e de nós intermédios que de T1s. Em média, cada nó 

			\noindent Outra estatística interessante a apresentar e a analisar seria o número de ASes pelas quais um pacote tem de viajar para ir de um nó até ao outro e quais os tipos de ligações que tem de percorrer até chegar ao seu destino. Tal implementação poderia ter sido feita no algoritmo que informa qual o melhor caminho a percorrer pela preferência de tipo de caminho. Podia, ainda, ser acrescentada a escolha do caminho mais curto, de forma a ser uma representação mais fiel do BGP, acrescentando a condição de que, para além de escolher o caminho preferencial, escolher o que apresentar o menor número número de saltos.

	\section{Conclusão}
	
\end{document}
